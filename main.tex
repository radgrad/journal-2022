%%%% Proceedings format for most of ACM conferences (with the exceptions listed below) and all ICPS volumes.
\documentclass[acmsmall]{acmart}
\usepackage{multirow}
\usepackage{listings}
\usepackage{subcaption}
\usepackage[T1]{fontenc}

\def\BibTeX{{\rm B\kern-.05em{\sc i\kern-.025em b}\kern-.08emT\kern-.1667em\lower.7ex\hbox{E}\kern-.125emX}}
\acmYear{2022} \acmVolume{1} \acmNumber{1} \acmArticle{1} \acmMonth{1}
\acmDOI{10.1145/3418301}
\acmISBN{123-4567-24-567/08/06}
\acmJournal{TOCE}
\acmYear{2022} \acmVolume{1} \acmNumber{1} \acmArticle{1} \acmMonth{1}
\copyrightyear{2022}
\acmPrice{15.00}

\begin{document}
\title[RadGrad: Results of a case study]{Improving engagement, diversity, and retention in computer science with RadGrad: Results of a case study}

\author{Philip M. Johnson}
\email{johnson@hawaii.edu}
\author{Carleton Moore}
\email{cmoore@hawaii.edu}
\affiliation{%
  \institution{Department of Information and Computer Sciences, University of Hawaii at Manoa}
  \city{Honolulu}
  \state{HI}
  \postcode{96822}
}

\author{Peter Leong}
\email{peterleo@hawaii.edu}
\author{Seungoh Paek}
\email{spaek@hawaii.edu}
\affiliation{%
  \institution{Department of Learning Design and Technology, University of Hawaii at Manoa}
  \city{Honolulu}
  \state{HI}
  \postcode{96822}
}
\orcid{0000-0002-9508-9225}

\renewcommand{\shortauthors}{Johnson et al.}

\newcommand{\AbstractCategory}[1]{%
  \par\addvspace{.5\baselineskip}% adjust to suit
  \noindent\textbf{#1}\quad\ignorespaces
}

\begin{abstract}

\AbstractCategory{Objectives}
RadGrad is a curriculum initiative implemented via a web-based application that combines features of social networks, degree planners, and serious games. RadGrad redefines the traditional meanings of ``progress'' and ``success'' in the undergraduate computer science degree program. The primary research question reported in this paper is: What is the impact of RadGrad on student engagement, diversity, and retention, and the opportunities and challenges that result from the use of the system?

\AbstractCategory{Participants}
To gain insight into our research question, this paper reports on data we obtained from major stakeholder groups, including: 498 students, 11 Faculty, and 2 Advisors.

\AbstractCategory{Study Methods}
We used a qualitative experimental research design, including data obtained through questionnaire responses from our major stakeholder group as well as data from system instrumentation.

\AbstractCategory{Findings}
We found strong evidence to support the hypothesis that RadGrad can have a positive impact on student engagement in Computer Science for students who are relatively new to the discipline. We found some evidence to support the hypothesis that RadGrad can have a positive impact on student diversity. We were unable to evaluate the impact of RadGrad on retention.

\AbstractCategory{Conclusions}
The data gathered in this study indicates that technologies like RadGrad provide a promising means to improve engagement, diversity, and (potentially) retention in Computer Science.  The data suggests that development and evaluation of RadGrad at other institutions and/or in other STEM disciplines would provide useful insight into the generality of the approach. It also indicates that faculty engagement with the technology would increase its effectiveness, but that obtaining faculty buy-in is challenging.

\end{abstract}

\begin{CCSXML}
<ccs2012>
<concept>
<concept_id>10003456.10003457.10003527.10003538</concept_id>
<concept_desc>Social and professional topics~Informal education</concept_desc>
<concept_significance>500</concept_significance>
</concept>
<concept>
<concept_id>10003456.10003457.10003527.10003539</concept_id>
<concept_desc>Social and professional topics~Computing literacy</concept_desc>
<concept_significance>500</concept_significance>
</concept>
</ccs2012>
\end{CCSXML}

\ccsdesc[500]{Social and professional topics~Informal education}
\ccsdesc[500]{Social and professional topics~Computing literacy}

\keywords{Curriculum Initiative; Diversity; Engagement; Retention}

\maketitle

\section{Introduction}

\subsection{Research Problem}
{\em\small Frame the problem or question and its context. Review, critique, and synthesize the applicable literature to identify key issues/debates/theoretical frameworks in the relevant literature to clarify barriers, knowledge gaps, or practical needs.}


\subsection{Research Goals}
{\em\small State the purpose(s)/goal(s)/aim(s) of the study. State the target audience, if specific. Provide the rationale for fit of design used to investigate this purpose/goal (e.g., theory building, explanatory, developing understanding, social action, description, highlighting social practices). Describe the approach to inquiry, if it illuminates the objectives and research rationale (e.g., descriptive, interpretive, feminist, psychoanalytic, postpositivist, critical, postmodern, constructivist, or pragmatic approaches).}

\section{Method}

\subsection{Research Design Overview}

{\em\small  Summarize the research design, including data-collection strategies, data-analytic strategies, and, if illuminating, approaches to inquiry (e.g., descriptive, interpretive, feminist, psychoanalytic, postpositivist, critical, postmodern, constructivist, or pragmatic approaches). Provide the rationale for the design selected.}

\subsection{Study Participants}

\subsubsection{Researcher Description}

{\em\small Describe the researchers’ backgrounds in approaching the study, emphasizing their prior understandings of the phenomena under study (e.g., interviewers, analysts, or research team). Describe how prior understandings of the phenomena under study were managed and/or influenced the research (e.g., enhancing, limiting, or structuring data collection and analysis).}

\subsubsection{Participants}

{\em\small Provide the numbers of participants/documents/events analyzed. Describe the demographics/cultural information, perspectives of participants, or characteristics of data sources that might influence the data collected. Describe existing data sources, if relevant (e.g., newspapers, internet, archive). Provide data repository information for openly shared data, if applicable. Describe archival searches or process of locating data for analyses, if applicable. }


\subsubsection{Researcher–Participant Relationship}

{\em\small  Describe the relationships and interactions between researchers and participants relevant to the research process and any impact on the research process (e.g., was there a relationship prior to research, are there any ethical considerations relevant to prior relationships). }


\subsection{Participant Recruitment}
\subsubsection{Recruitment Process}

{\em\small Describe the recruitment process (e.g., face-to-face, telephone, mail, email) and any recruitment protocols. Describe any incentives or compensation, and provide assurance of relevant ethical processes of data collection and consent process as relevant (may include institutional review board approval, particular adaptations for vulnerable populations, safety monitoring). Describe the process by which the number of participants was determined in relation to the
study design. Provide any changes in numbers through attrition and final number of participants/sources (if relevant, refusal rates or reasons for dropout). Describe the rationale for decision to halt data collection (e.g., saturation). Convey the study purpose as portrayed to participants, if different from the purpose stated. }

\subsubsection{Participant Selection}

{\em\small Describe the participants/data source selection process (e.g., purposive sampling methods, such as maximum variation; convenience sampling methods, such as snowball selection; theoretical sampling; diversity sampling) and inclusion/exclusion criteria. Provide the general context for the study (when data were collected, sites of data collection). If your participant selection is from an archived data set, describe the recruitment and selection process from that data set as well as any decisions in selecting sets of participants from that data set. }

\subsection{Data Collection}

\subsubsection{Data Collection/Identification Procedures}

{\em\small State the form of data collected (e.g., interviews, questionnaires, media, observation). Describe the origins or evolution of the data-collection protocol. Describe any alterations of data-collection strategy in response to the evolving findings or the study rationale. Describe the data-selection or data-collection process (e.g., were others present when data were collected, number of times data were collected, duration of collection, context). Convey the extensiveness of engagement (e.g., depth of engagement, time intensiveness of data collection). For interview and written studies, indicate the mean and range of the time duration in the data-collection process (e.g., interviews were held for 75 to 110 min, with an average interview time of 90 min). Describe the management or use of reflexivity in the data-collection process, as it
illuminates the study. Describe questions asked in data collection: content of central questions, form of questions (e.g., open vs. closed). }

\subsubsection{Recording and Data Transformation}

{\em\small Identify data audio/visual recording methods, field notes, or transcription processes used.}

\subsection{Analysis}

\subsubsection{Data-Analytic Strategies}

{\em\small Describe the methods and procedures used and for what purpose/goal. Explicate in detail the process of analysis, including some discussion of the procedures (e.g., coding, thematic analysis) following a principle of transparency. Describe coders or analysts and their training, if not already described in the researcher
description section (e.g., coder selection, collaboration groups). Identify whether coding categories emerged from the analyses or were developed a priori. Identify units of analysis (e.g., entire transcript, unit, text) and how units were formed, if applicable. Describe the process of arriving at an analytic scheme, if applicable (e.g., if one was
developed before or during the analysis or was emergent throughout). Provide illustrations and descriptions of the analytic scheme development, if relevant. Indicate software, if used. }

\subsubsection{Methodological Integrity}

{\em\small Demonstrate that the claims made from the analysis are warranted and have produced findings with methodological integrity. The procedures that support methodological integrity (i.e., fidelity and utility) typically are described across the relevant sections of a paper, but they could be addressed in a separate section when elaboration or emphasis would be helpful. Issues of methodological integrity include the following: Assess the adequacy of the data in terms of its ability to capture forms of diversity most relevant to the question, research goals, and inquiry approach. Describe how the researchers’ perspectives were managed in both the data collection and analysis (e.g., to limit their effect on the data collection, to structure the analysis). Demonstrate that findings are grounded in the evidence (e.g., using quotes, excerpts, or descriptions of researchers’ engagement in data collection). Demonstrate that the contributions are insightful and meaningful (e.g., in relation to the current literature and the study goal). Provide relevant contextual information for findings (e.g., setting of study, information about participants, interview question asked is presented before excerpt as needed). Present findings in a coherent manner that makes sense of contradictions or disconfirming evidence in the data (e.g., reconcile discrepancies, describe why a conflict might exist in the findings). Demonstrate consistency with regard to the analytic processes (e.g., analysts may use demonstrations of analyses to support consistency, describe their development of a stable perspective, interrater reliability, consensus) or describe responses to inconsistencies, as relevant (e.g., coders switching midway through analysis, an interruption in the analytic process). If alterations in methodological integrity were made for ethical reasons, explicate those reasons and the adjustments made. Describe how support for claims was supplemented by any checks added to the qualitative analysis. Examples of supplemental checks that can strengthen the research may include: transcripts/data collected returned to participants for feedback triangulation across multiple sources of information, findings, or investigators; checks on the interview thoroughness or interviewer demands; consensus or auditing process; member checks or participant feedback on findings; data displays/matrices; in-depth thick description, case examples, or illustrations; structured methods of researcher reflexivity (e.g., sending memos, field notes, diary, logbooks, journals, bracketing); checks on the utility of findings in responding to the study problem (e.g., an evaluation of whether a solution worked) }

\section{Findings}

{\em\small Describe research findings (e.g., themes, categories, narratives) and the meaning and understandings that the researcher has derived from the data analysis. Present research findings in a way that is compatible with the study design. Demonstrate the analytic process of reaching findings (e.g., quotes, excerpts of data). Present synthesizing illustrations (e.g., diagrams, tables, models), if useful in organizing and conveying findings. Photographs or links to videos can be used. }

\section{Discussion}

\subsection{Central Contributions to the Discipline}

{\em\small Describe the central contributions and their significance in advancing disciplinary understandings.}

\subsection{Types of Contributions Made by Findings}

{\em\small Describe the types of contributions made by findings (e.g., challenging, elaborating on, and supporting prior research or theory in the literature describing the relevance) and how findings can be best utilized. }

\subsection{Comparing Prior Theories and Research Findings}

{\em\small Identify similarities and differences from prior theories and research findings.}

\subsection{Alternative Explanations of Findings}

{\em\small Reflect on any alternative explanations of the findings.}

\subsection{Strengths and Limitations}

{\em\small Identify the study’s strengths and limitations (e.g., consider how the quality, source, or types of the data or the analytic processes might support or weaken its methodological integrity).}

\subsection{Limitations on Transferability}

{\em\small Describe the limits of the scope of transferability (e.g., what should readers bear in mind when using findings across contexts). }

\subsection{Ethical Dilemmas/Challenges}

{\em\small Revisit any ethical dilemmas or challenges that were encountered, and provide related suggestions for future researchers. }

\subsection{Implications for Future}

{\em\small Consider the implications for future research, policy, or practice.}

\section{Acknowledgements}

{\em\small Acknowledgments are placed before the references. Add information about grants, awards, or other types of funding that you have received to support your research. This information must be anonymized in the version of the paper you submit for review. }

\bibliographystyle{ACM-Reference-Format}
\bibliography{bibliography.bib}

\appendix


\end{document}
