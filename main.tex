%%%% Proceedings format for most of ACM conferences (with the exceptions listed below) and all ICPS volumes.
\documentclass[acmsmall]{acmart}
\usepackage{multirow}
\usepackage{listings}
\usepackage{subcaption}
\usepackage[T1]{fontenc}

\def\BibTeX{{\rm B\kern-.05em{\sc i\kern-.025em b}\kern-.08emT\kern-.1667em\lower.7ex\hbox{E}\kern-.125emX}}
\acmYear{2022} \acmVolume{1} \acmNumber{1} \acmArticle{1} \acmMonth{1}
\acmDOI{10.1145/3418301}
\acmISBN{123-4567-24-567/08/06}
\acmJournal{TOCE}
\acmYear{2022} \acmVolume{1} \acmNumber{1} \acmArticle{1} \acmMonth{1}
\copyrightyear{2022}
\acmPrice{15.00}

\begin{document}
\title[RadGrad: Results of a case study]{Improving engagement, diversity, and retention in computer science with RadGrad: Results of a case study}

\author{Philip M. Johnson}
\email{johnson@hawaii.edu}
\author{Carleton Moore}
\email{cmoore@hawaii.edu}
\affiliation{%
  \institution{Department of Information and Computer Sciences, University of Hawaii at Manoa}
  \city{Honolulu}
  \state{HI}
  \postcode{96822}
}

\author{Peter Leong}
\email{peterleo@hawaii.edu}
\author{Seungoh Paek}
\email{spaek@hawaii.edu}
\affiliation{%
  \institution{Department of Learning Design and Technology, University of Hawaii at Manoa}
  \city{Honolulu}
  \state{HI}
  \postcode{96822}
}
\orcid{0000-0002-9508-9225}

\renewcommand{\shortauthors}{You et al.}

%Stolen from: https://tex.stackexchange.com/questions/74371/subdividing-structured-abstracts
\newcommand{\AbstractCategory}[1]{%
  \par\addvspace{.5\baselineskip}% adjust to suit
  \noindent\textbf{#1}\quad\ignorespaces
}

%Structured abstracts are encouraged when possible. Please indicate the mixed methods design, including types of participants or data sources, analytic strategy, main results/findings, and major implications/significance.
%
% The abstract is a short summary of the work to be presented in the article.
\begin{abstract}
\AbstractCategory{Objectives}This is the start of the structured abstract, which is encouraged (but not required) for all TOCE papers. State the problem under investigation, including main hypotheses.


\AbstractCategory{Participants}
Describe subjects participants, specifying their pertinent characteristics for the study. Participants are described in greater detail in the body of the paper.

\AbstractCategory{Study Methods} Describe the study method, including‒research design (e.g., experiment, observational study)‒sample size‒materials used (e.g., instruments, apparatus)‒outcome measures‒data-gathering procedures, including a brief description of the source of any secondary data. If the study is a secondary data analysis, so indicate.

\AbstractCategory{Findings} Report findings, including effect sizes and confidence intervals or statistical significance levels.

\AbstractCategory{Conclusions} State conclusions, beyond just results, and report the implications or applications.

\end{abstract}



%
% The code below is generated by the tool at http://dl.acm.org/ccs.cfm.
% Please copy and paste the code instead of the example below.


\begin{CCSXML}
<ccs2012>
<concept>
<concept_id>10002950.10003705.10003708</concept_id>
<concept_desc>Mathematics of computing~Statistical software</concept_desc>
<concept_significance>500</concept_significance>
</concept>
<concept>
<concept_id>10010405.10010489</concept_id>
<concept_desc>Applied computing~Education</concept_desc>
<concept_significance>500</concept_significance>
</concept>
<concept>
<concept_id>10003120.10003121.10011748</concept_id>
<concept_desc>Human-centered computing~Empirical studies in HCI</concept_desc>
<concept_significance>300</concept_significance>
</concept>
<concept>
<concept_id>10011007.10011006.10011050.10011017</concept_id>
<concept_desc>Software and its engineering~Domain specific languages</concept_desc>
<concept_significance>300</concept_significance>
</concept>
</ccs2012>
\end{CCSXML}

\ccsdesc[500]{Mathematics of computing~Statistical software}
\ccsdesc[500]{Applied computing~Education}
\ccsdesc[300]{Human-centered computing~Empirical studies in HCI}
\ccsdesc[300]{Software and its engineering~Domain specific languages}


%
% Keywords. The author(s) should pick words that accurately describe the work being
% presented. Separate the keywords with commas.
\keywords{comma, separated}


%
% This command processes the author and affiliation and title information and builds
% the first part of the formatted document.
\maketitle

%%Optional
\textbf{Author's Note:} Provide acknowledgment and explanation of any special circumstances, including:

\begin{itemize}
    \item Registration information if the study has been registered
    \item Use of data also appearing in previous publications
    \item Prior reporting of the fundamental data in dissertations or conference papers
    \item Sources of funding or other support‒relationships or affiliations that may be perceived as conflicts of interest
    \item Previous (or current) affiliation of authors if different from location where the study  was conducted
    \item Contact information for the corresponding author
    \item Additional information of importance to the reader that may not be appropriately included in other sections of the paper
    \item TOCE uses a double-blind review process. Omit author information and all identifying information in papers submitted for review. Please see \hyperlink{https://dl.acm.org/journal/toce/author-guidelines}{\color{blue}{TOCE’s Author Guidelines page}} for advice on anonymizing your paper.
\end{itemize}


\section{Introduction}
ACM TOCE strongly encourages authors to submit papers that adhere to the \hyperlink{https://apastyle.apa.org/jars/}{\color{blue}{American Psychological Association’s Journal Article Reporting Standards}}  (JARS).  Some papers that lie within the scope of ACM TOCE are not a good fit for these standards; we welcome such submissions. Whether or not your paper is a good fit for the standards, they can provide useful guidance to help make your paper more accessible to reviewers and the TOCE audience.
Here, we provide a JARS template for \hyperlink{https://apastyle.apa.org/jars/mixed-table-1.pdf}{\color{blue}{mixed-method research designs}}.  See also the TOCE JARS templates for \hyperlink{https://apastyle.apa.org/jars/quantitative}{\color{blue}{quantitative}} and \hyperlink{https://apastyle.apa.org/jars/qualitative}{\color{blue}{qualitative}} research designs.

Note that the ACM provides an article templates for \hyperlink{https://www.acm.org/binaries/content/assets/publications/taps/acm_submission_template.docx}{\color{blue}{Microsoft Word}} and \hyperlink{https://www.acm.org/binaries/content/assets/publications/consolidated-tex-template/acmart-master.zip}{\color{blue}{Latex}}. Please download it and reference as appropriate as you write your paper. They contain detailed advice and guidance on a range of specific formatting issues not covered in this template.

\textit{Guidance for Authors:} This section may convey barriers in the literature that suggest a need for both qualitative and quantitative data.

\textit{Guidance for Reviewers:} Theory or conceptual framework use in mixed methods varies depending on the  specific mixed methods design or procedures used. Theory may be used inductively or deductively (or both) in mixed methods research

\subsection{Description of Research Problem or Question}

State the importance of the problem, including theoretical or practical implications. Frame the problem or question and its context. Review, critique, and synthesize the applicable literature to identify key issues/debates/ theoretical frameworks in the relevant literature to clarify barriers, knowledge gaps, or practical needs.

\subsection{Study Objectives/Aims/Research Goals}
\begin{itemize}
\item State the purpose(s)/goal(s)/aim(s) of the study. Order these to reflect the type of mixed methods design used.
\item State the target audience, if specific.
\item Provide the rationale for fit of design used to investigate this purpose/goal (e.g., theory building, explanatory, developing understanding, social action, description, highlighting social practices).
\item Describe the ways approaches to inquiry were combined, as it illuminates the objectives and mixed methods rationale (e.g., descriptive, interpretive, feminist, psychoanalytic, postpositivist, critical, postmodern, constructivist, or pragmatic approaches).
\end{itemize}

\textit{Guidance for Authors:} If relevant to objectives, explain the relation of the current analysis to prior articles/ publications.

\textit{Guidance for Reviewers:} This subsection should describe the results to be obtained from using the mixed methods design type where “mixing” or integration occurs (e.g., the aim is to explain quantitative survey results with qualitative interviews in an explanatory sequential design). For instance, the goal of a qualitative phase could be the development of a conceptual model, the goal of a quantitative phase could be hypothesis testing based upon that model, and the goal of the mixed methods phase could be to generate integrated support for a theory based upon quantitative and qualitative evidence

\section{Method}

See the \hyperlink{https://apastyle.apa.org/jars/quantitative}{\color{blue}{JARS-QUANT}} and \hyperlink{https://apastyle.apa.org/jars/qualitative}{\color{blue}{JARS-QUAL}} standards for guidance on this section.

\subsection{Research Design Overview}

\begin{itemize}
\item Summarize the research design, including data-collection strategies, data-analytic strategies, and, if illuminating, approaches to inquiry (e.g., descriptive, interpretive, feminist, psychoanalytic, postpositivist, critical, postmodern, constructivist, or pragmatic approaches).
\item Explain why mixed methods research is appropriate as a methodology given the paper’s goals.
\item Identify the type of mixed methods design used and define it.
\item Indicate the qualitative approach to inquiry and the quantitative design used within the mixed methods design type (e.g., ethnography, randomized experiment).
\item If multiple approaches to inquiry were combined, describe how this was done and provide a rationale (e.g., descriptive, interpretive, feminist, psychoanalytic, postpositivist, critical, postmodern, constructivist, or pragmatic approaches), as it is illuminating for the mixed method in use.
\item Provide a rationale or justification for the need to collect both qualitative and quantitative data and the added value of integrating the results (findings) from the two databases.
\end{itemize}

\textit{Guidance for Reviewers:}
\begin{itemize}
\item It may be helpful to provide a definition of the mixed-methods approach from a major reference in the field.
\item Mixed methods research involves rigorous methods, both qualitative and quantitative. Refer to the See the JARS–Qual (qualitative) and JARS–Quant (quantitative) Standards for details of rigor.
\item One of the most widely discussed topics in the mixed methods literature is research designs. There is not a generic mixed methods design but rather multiple types of designs. Basic, core designs include convergent design, explanatory sequential design, and exploratory sequential design. Although the names and types of designs may differ among mixed methods writers, a common understanding is that procedures for conducting a mixed methods study may differ from one project to another. Further, these basic procedures can be expanded by linking mixed methods to other designs (e.g., an intervention or experimental trial mixed methods study), to theories or standpoints (e.g., a feminist mixed methods study), or to other methodologies (e.g., a participatory action research mixed methods study)
\end{itemize}


\subsection{Participants or Other Data Sources}

\begin{itemize}
\item Provide the numbers of participants/documents/events analyzed.
\item When data are collected from multiple sources, clearly identify the sources of qualitative and quantitative data (e.g., participants, text), their characteristics, and the relationship between the data sets, if there is one (e.g., an embedded design).
\item State the data sources in the order of procedures used in the design type (e.g., qualitative sources first in an exploratory sequential design followed by quantitative sources), if a sequenced design is used in the mixed methods study
\item Describe the demographics/cultural information, perspectives of participants, or characteristics of data sources that might influence the data collected.
\item Describe existing data sources, if relevant (e.g., newspapers, internet, archive).
\item Provide data repository information for openly shared data, if applicable.
\item Describe archival searches or process of locating data for analyses, if applicable.
\end{itemize}

\textit{Guidance for Authors:}
\begin{itemize}
\item Because multiple sources of data are collected, separate descriptions of samples are needed when they differ. A table of qualitative sources and quantitative sources is helpful. This table could include type of data, when data were collected, and from whom. This table might also include study aims/research questions for each data source and anticipated outcomes of the study. In mixed methods research, this table is often called an "implementation matrix."
\item Rather than describe data as represented in numbers versus words, it is better to describe sources of data as open-ended information (e.g., qualitative interviews) and closed-ended information (e.g., quantitative instruments).
\end{itemize}

\textit{Researcher Description:}
\begin{itemize}
\item Describe the researchers’ backgrounds in approaching the study, emphasizing their prior understandings of the phenomena under study (e.g., interviewers, analysts, or research team).
\item Describe how prior understandings of the phenomena under study were managed and/or influenced the research (e.g., enhancing, limiting, or structuring data collection and analysis).
\end{itemize}

\textit{Guidance for Authors: }
Because mixed methods research includes qualitative research, and reflexivity is often included in qualitative research, we recommend statements as to how the researchers’ backgrounds influence the research.
\textit{Guidance for Reviewers:}
It is helpful to establish in a publication the researchers’ experiences (or research teams’ experiences) with both qualitative and quantitative research as a prerequisite for conducting mixed methods research




\subsection{Participant Recruitment}

\subsubsection{Participant Sampling or Selection}
\begin{itemize}
\item See the \hyperlink{https://apastyle.apa.org/jars/quantitative}{\color{blue}{JARS-QUANT}} and \hyperlink{https://apastyle.apa.org/jars/qualitative}{\color{blue}{JARS-QUAL}} standards for guidance.
\item Describe the qualitative and the quantitative sampling in separate sections.
\item Relate the order of the sections to the procedures used in the mixed methods design type.
\end{itemize}

\subsubsection{Participant Selection}
\begin{itemize}
\item See the \hyperlink{https://apastyle.apa.org/jars/quantitative}{\color{blue}{JARS-QUANT}} and \hyperlink{https://apastyle.apa.org/jars/qualitative}{\color{blue}{JARS-QUAL}} standards for guidance.
\item Discuss the recruitment strategy for qualitative and quantitative research separately.
\end{itemize}

\subsection{Data Collection}

\subsubsection{Data Collection/Identification Procedures}

See the \hyperlink{https://apastyle.apa.org/jars/quantitative}{\color{blue}{JARS-QUANT}} and \hyperlink{https://apastyle.apa.org/jars/qualitative}{\color{blue}{JARS-QUAL}} standards for guidance.

\subsubsection{Recording and Data Transformation}

Identify data audio/visual recording methods, field notes, or transcription processes used.

\subsection{Data Analysis}
\begin{itemize}
\item See the \hyperlink{https://apastyle.apa.org/jars/quantitative}{\color{blue}{JARS-QUANT}} and \hyperlink{https://apastyle.apa.org/jars/qualitative}{\color{blue}{JARS-QUAL}} standards for guidance.
\item Devote separate sections to the qualitative data analysis, the quantitative data analysis, and the mixed methods analysis. This mixed methods analysis consists of ways that the quantitative and qualitative results will be “mixed” or integrated according to the type of mixed methods design used (e.g., merged in a convergent design, connected in explanatory sequential designs and in exploratory sequential designs).
\end{itemize}

\subsection{Validity, Reliability, and Methodological Integrity}
\begin{itemize}
\item See the \hyperlink{https://apastyle.apa.org/jars/quantitative}{\color{blue}{JARS-QUANT}} and \hyperlink{https://apastyle.apa.org/jars/qualitative}{\color{blue}{JARS-QUAL}} standards for guidance.
\item Indicate methodological integrity, quantitative validity and reliability, and mixed methods validity or legitimacy. Further assessments of mixed methods integrity are also indicated to show the quality of the research process and the inferences drawn from the intersection of the quantitative and qualitative data.
\end{itemize}

\section{FINDINGS/RESULTS}
\begin{itemize}
\item See the \hyperlink{https://apastyle.apa.org/jars/quantitative}{\color{blue}{JARS-QUANT}} and \hyperlink{https://apastyle.apa.org/jars/qualitative}{\color{blue}{JARS-QUAL}} standards for guidance.
\item Indicate how the qualitative and quantitative results were “mixed” or integrated (e.g., discussion; tables of joint displays; graphs; data transformation in which one form of data is transformed to the other, such as qualitative text, codes, themes are transformed into quantitative counts or variables).
\end{itemize}

\textit{Guidance for Authors:}  In mixed methods research, the Findings section typically includes sections on qualitative findings, quantitative results, and mixed methods results. This section should mirror the type of mixed methods design in terms of sequence (i.e., whether quantitative strand or qualitative strand comes first; if both are gathered at the same time, either qualitative findings or quantitative results could be presented first).

\textit{Guidance for Reviewers:} In mixed methods Results sections (or in the Discussion section to follow), authors are conveying their mixed methods analysis through “joint display” tables or graphs that array the qualitative results (e.g., themes) against the quantitative results (e.g., categorical or continuous data). This enables researchers to directly compare results or to see how results differ between the quantitative and qualitative strands.

\section{DISCUSSION}
See the \hyperlink{https://apastyle.apa.org/jars/quantitative}{\color{blue}{JARS-QUANT}} and \hyperlink{https://apastyle.apa.org/jars/qualitative}{\color{blue}{JARS-QUAL}} standards for guidance.

\textit{Guidance for Authors:} Typically, the Discussion section, like the Method and Findings/Results, mirrors in sequence the procedures used in the type of mixed methods design. It also reflects on the implications of the integrated findings from across the two methods.

\section{ACKNOWLEDGMENTS}
Acknowledgments are placed before the references. Add information about grants, awards, or other types of funding that you have received to support your research. This information must be anonymized in the version of the paper you submit for review.

\bibliographystyle{ACM-Reference-Format}
\bibliography{bibliography.bib}

\appendix


\end{document}
