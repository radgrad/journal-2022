%%%% Proceedings format for most of ACM conferences (with the exceptions listed below) and all ICPS volumes.
\documentclass[acmsmall]{acmart}
\usepackage{multirow}
\usepackage{listings}
\usepackage{subcaption}
\usepackage[T1]{fontenc}
\usepackage{color}

\def\BibTeX{{\rm B\kern-.05em{\sc i\kern-.025em b}\kern-.08emT\kern-.1667em\lower.7ex\hbox{E}\kern-.125emX}}
    
    
% Rights management information. 
% This information is sent to you when you complete the rights form.
% These commands have SAMPLE values in them; it is your responsibility as an author to replace
% the commands and values with those provided to you when you complete the rights form.
%
% These commands are for a PROCEEDINGS abstract or paper.
%\copyrightyear{2018}
%\acmYear{2018}
%\setcopyright{acmlicensed}
%\acmConference[Woodstock '18]{Woodstock '18: ACM Symposium on Neural Gaze Detection}{June 03--05, 2018}{Woodstock, NY}
%\acmBooktitle{Woodstock '18: ACM Symposium on Neural Gaze Detection, June 03--05, 2018, Woodstock, NY}
%\acmPrice{15.00}
%\acmDOI{10.1145/1122445.1122456}
%\acmISBN{978-1-4503-9999-9/18/06}

\acmYear{2021} \acmVolume{1} \acmNumber{1} \acmArticle{1} \acmMonth{1} 
% DOI
\acmDOI{10.1145/3418301}

% ISBN
\acmISBN{123-4567-24-567/08/06}

%Journal
\acmJournal{TOCE}

\acmYear{2021} \acmVolume{1} \acmNumber{1} \acmArticle{1} \acmMonth{1} 

\copyrightyear{2021}
\acmPrice{15.00}


\begin{document}


%From the APA JARS Specification
%Identify main variables and theoretical issues under investigation and the relationships between them.
%Identify the populations studied.
\title[JARS Quantitative Template for ACM TOCE]{The JARS Quantitative Template for ACM TOCE}

%TOCE uses a double-blind review process. Omit author information and all identifying information in papers submitted for review. Please see TOCE’s Author Guidelines page (https://dl.acm.org/journal/toce/author-guidelines) for advice on anonymizing your paper.
\author{Obiwan Kenobi}
\email{youremail@galactic.republic.edu}
\affiliation{\institution{University of Nevada, Coruscant}}
\orcid{0000-0002-9508-9225}


\renewcommand{\shortauthors}{You et al.}

%Stolen from: https://tex.stackexchange.com/questions/74371/subdividing-structured-abstracts
\newcommand{\AbstractCategory}[1]{%
  \par\addvspace{.5\baselineskip}% adjust to suit
  \noindent\textbf{#1}\quad\ignorespaces
}

%
% The abstract is a short summary of the work to be presented in the article.
\begin{abstract}
\AbstractCategory{Objectives}State the problem under investigation, including main hypotheses.


\AbstractCategory{Participants}
Describe participants, specifying their pertinent characteristics for the study. Participants are described in greater detail in the body of the paper.

\AbstractCategory{Study Methods} Describe the study method, including‒research design (e.g., experiment, observational study)‒sample size‒materials used (e.g., instruments, apparatus)‒outcome measures‒data-gathering procedures, including a brief description of the source of any secondary data. If the study is a secondary data analysis, so indicate.

\AbstractCategory{Findings} Report findings, including effect sizes and confidence intervals or statistical significance levels.

\AbstractCategory{Conclusions} State conclusions, beyond just results, and report the implications or applications.

\end{abstract}



%
% The code below is generated by the tool at http://dl.acm.org/ccs.cfm.
% Please copy and paste the code instead of the example below.


\begin{CCSXML}
<ccs2012>
<concept>
<concept_id>10002950.10003705.10003708</concept_id>
<concept_desc>Mathematics of computing~Statistical software</concept_desc>
<concept_significance>500</concept_significance>
</concept>
<concept>
<concept_id>10010405.10010489</concept_id>
<concept_desc>Applied computing~Education</concept_desc>
<concept_significance>500</concept_significance>
</concept>
<concept>
<concept_id>10003120.10003121.10011748</concept_id>
<concept_desc>Human-centered computing~Empirical studies in HCI</concept_desc>
<concept_significance>300</concept_significance>
</concept>
<concept>
<concept_id>10011007.10011006.10011050.10011017</concept_id>
<concept_desc>Software and its engineering~Domain specific languages</concept_desc>
<concept_significance>300</concept_significance>
</concept>
</ccs2012>
\end{CCSXML}

\ccsdesc[500]{Mathematics of computing~Statistical software}
\ccsdesc[500]{Applied computing~Education}
\ccsdesc[300]{Human-centered computing~Empirical studies in HCI}
\ccsdesc[300]{Software and its engineering~Domain specific languages}


%
% Keywords. The author(s) should pick words that accurately describe the work being
% presented. Separate the keywords with commas.
\keywords{comma, separated}


%
% This command processes the author and affiliation and title information and builds
% the first part of the formatted document.
\maketitle

%%Optional
\vspace{3mm}
\textbf{Author's Note:} Provide acknowledgment and explanation of any special circumstances, including:

\begin{itemize}
    \item Registration information if the study has been registered
    \item Use of data also appearing in previous publications
    \item Prior reporting of the fundamental data in dissertations or conference papers
    \item Sources of funding or other support‒relationships or affiliations that may be perceived as conflicts of interest
    \item Previous (or current) affiliation of authors if different from location where the study  was conducted
    \item Contact information for the corresponding author
    \item Additional information of importance to the reader that may not be appropriately included in other sections of the paper
    \item TOCE uses a double-blind review process. Omit author information and all identifying information in papers submitted for review. Please see \hyperlink{https://dl.acm.org/journal/toce/author-guidelines}{\color{blue}{TOCE’s Author Guidelines page}} for advice on anonymizing your paper.
\end{itemize}


\section{Introduction}

ACM TOCE strongly encourages authors to submit papers that adhere to the \hyperlink{https://apastyle.apa.org/jars/}{\color{blue}{American Psychological Association’s Journal Article Reporting Standards}}  (JARS). Some papers that lie within the scope of ACM TOCE are not a good fit for these standards; we welcome such submissions. Whether or not your paper is a good fit for the standards, they can provide useful guidance to help make your paper more accessible to reviewers and the TOCE audience. Here, we provide here a JARS template for \hyperlink{https://apastyle.apa.org/jars/quant-table-1.pdf}{\color{blue}{general quantitative research designs}}. Use \hyperlink{https://apastyle.apa.org/jars/jars-quant-participant-flowchart.pdf}{\color{blue}{this flowchart}} to identify the JARS reporting guidelines for your specific quantitative design, including \hyperlink{https://apastyle.apa.org/jars/quant-table-2.pdf}{\color{blue}{Experimental Designs}}, \hyperlink{https://apastyle.apa.org/jars/quant-table-2a.pdf}{\color{blue}{Random Assignment}}, \hyperlink{https://apastyle.apa.org/jars/quant-table-2b.pdf}{\color{blue}{Non-Random Assignment}}, \hyperlink{https://apastyle.apa.org/jars/quant-table-2c.pdf}{\color{blue}{Clinical Trials}}, \hyperlink{https://apastyle.apa.org/jars/quant-table-3.pdf}{\color{blue}{Non-Experimental Designs}}, Special Designs (\hyperlink{https://apastyle.apa.org/jars/quant-table-4.pdf}{\color{blue}{Longitudinal Studies}}, \hyperlink{https://apastyle.apa.org/jars/quant-table-5.pdf}{\color{blue}{N-of-1 Studies}}, \hyperlink{https://apastyle.apa.org/jars/quant-table-6.pdf}{\color{blue}{Replication Studies}}), Analytic Methods (\hyperlink{https://apastyle.apa.org/jars/quant-table-7.pdf}{\color{blue}{Structural Equation Modeling}}, \hyperlink{https://apastyle.apa.org/jars/quant-table-8.pdf}{\color{blue}{Bayesian Statistics}}), and \hyperlink{https://apastyle.apa.org/jars/quant-table-9.pdf}{\color{blue}{Meta-Analyses}}. The reporting standards for these more specific quantitative designs can be readily integrated into this template. See also the TOCE JARS templates for \hyperlink{https://apastyle.apa.org/jars/qualitative}{\color{blue}{qualitative}} and \hyperlink{https://apastyle.apa.org/jars/mixed-methods}{\color{blue}{mixed}} research designs.


Note that the ACM provides \hyperlink{https://www.acm.org/binaries/content/assets/publications/consolidated-tex-template/acmart-master.zip}{\color{blue}{an article template for LateX}}. Please download it and reference as appropriate as you write your paper. It contains detailed advice and guidance on a range of specific formatting issues not covered in this template.


\subsection{Problem}

State the importance of the problem, including theoretical or practical implications.

\subsection{Review of Relevant Scholarship}

Provide a succinct review of relevant scholarship, including: 

\begin{itemize}
    \item Relation to previous work
    \item Differences between the current report and earlier reports if some aspects of this study have been reported on previously
\end{itemize}

\subsection{Hypothesis, Aims, and Objectives}

State specific hypotheses, aims, and objectives, including:
\begin{itemize}
    \item theories or other means used to derive hypotheses
    \item primary and secondary hypotheses‒other planned analyses
\end{itemize}

State how hypotheses and research design relate to one another.

\section{Method}

\subsection{Inclusion and Exclusion}
Report inclusion and exclusion criteria, including any restrictions based on demographic characteristics.

\subsection{Participant Characteristics}
Report major demographic characteristics (e.g., age, sex, ethnicity, socioeconomic status) and important topic-specific characteristics (e.g., achievement level in studies of educational interventions).

% I trust no one needs this in CS Ed Research! Commenting this out ~ Monica
%In the case of animal research, report the genus, species, and strain number or other specific identification, such as the name and location of the supplier and the stock designation. Give the number of animals and the animals’ sex, age, weight, physiological condition, genetic modification status, genotype, health–immune status, drug or test naïveté, and previous procedures to which the animal may have been subjected.

\subsection{Sampling Procedures}

\begin{itemize}
    \item Describe procedures for selecting participants, including:
    \begin{itemize}
        \item Sampling method if a systematic sampling plan was implemented
        \item Percentage of sample approached that actually participated
        \item Whether self-selection into the study occurred (either by individuals or by units, such as schools or clinics)
    \end{itemize}
    \item Describe settings and locations where data were collected as well as dates of data collection.
    \item Describe agreements and payments made to participants.
    \item Describe institutional review board agreements, ethical standards met, and safety monitoring.
\end{itemize}

\subsection{Sample Size, Power, and Precision}
Describe the sample size, power, and precision, including:
\begin{itemize}
    \item Intended sample size
    \item Achieved sample size, if different from the intended sample size
    \item Determination of sample size, including power analysis, or methods used to determine precision of parameter estimates and explanation of any interim analyses and stopping rules employed
\end{itemize}

\subsection{Measures and Covariates}
Define all primary and secondary measures and covariates, including measures collected but not included in the report.

\subsection{Data Collection}
Describe methods used to collect data.

\subsection{Quality of Measurements}
Describe methods used to enhance the quality of measurements, including‒training and reliability of data collectors‒use of multiple observations

\subsection{Instrumentation}
Provide information on validated or ad hoc instruments created for individual studies, for individual studies (e.g., psychometric and biometric properties)

\subsection{Masking}
Report whether participants, those administering the experimental manipulations, and those assessing the outcomes were aware of condition assignments. If masking took place, provide a statement regarding how it was accomplished and whether and how the success of masking was evaluated. 

\subsection{Psychometrics}
\begin{itemize}
    \item Estimate and report values of reliability coefficients for the scores analyzed (i.e., the researcher’s sample), if possible. 
    \item Provide estimates of convergent and discriminant validity where relevant.
    \item Report estimates related to the reliability of measures, including: 
    \begin{itemize}
        \item interrater reliability for subjectively scored measures and ratings‒test–retest coefficients in longitudinal studies in which the retest interval corresponds to the measurement schedule used in the study
        \item internal consistency coefficients for composite scales in which these indices are appropriate for understanding the nature of the instruments being used in the study
    \end{itemize}
    \item Report the basic demographic characteristics of other samples if reporting reliability or validity coefficients from those samples, such as those described in test manuals or in norming information for the instrument.
\end{itemize}

\subsection{Conditions and Design}

\begin{itemize}
    \item State whether conditions were manipulated or naturally observed. Report the type of design as per the JARS–Quant tables:
    \begin{itemize}
        \item experimental manipulation with participants randomized›   Table 2 and Module A
        \item experimental manipulation without randomization›   Table 2 and Module B
        \item clinical trial with randomization›   Table 2 and Modules A and C
        \item clinical trial without randomization›   Table 2 and Modules B and C
        \item nonexperimental design (i.e., no experimental manipulation): observational design, epidemiological design, natural history, and so forth (single-group designs or multiple-group comparisons)›   Table 3
        \item longitudinal design›   Table 4
        \item N-of-1 studies›   Table 5
        \item replications›   Table 6
    \end{itemize}

    \item Report the common name given to designs not currently covered in JARS–Quant.
\end{itemize}

\subsection{Data Diagnostics}

Describe planned data diagnostics, including‒criteria for post-data-collection exclusion of participants, if any‒criteria for deciding when to infer missing data and methods used for imputation of missing data‒definition and processing of statistical outliers‒analyses of data distributions‒data transformations to be used, if any

\subsection{Analytic Strategy}

Describe the analytic strategy for inferential statistics and protection against experiment-wise error for:
\begin{itemize}
    \item primary hypotheses
    \item secondary hypotheses
    \item exploratory hypotheses
\end{itemize}

\section{Results}

\subsection{Participant Flow}
Report the flow of participants, including:
\begin{itemize}
    \item total number of participants in each group at each stage of the study
    \item flow of participants through each stage of the study (include figure depicting flow, when possible; see the \hyperlink{https://apastyle.apa.org/jars/jars-quant-participant-flowchart.pdf}{\color{blue}{JARS–Quant Participant Flowchart}})
\end{itemize}

\subsection{Recruitment}
Provide dates defining the periods of recruitment and repeated measures or follow-up.

\subsection{ Statistics and Data Analysis}
Provide information detailing the statistical and data-analytic methods used, including:

%Monica - I don't think all the breaks in the below are correct, just tried it a bit to be able to read it better for now.
\begin{itemize}
    \item Missing data (frequency or percentages of missing data)
    \item Empirical evidence and/or theoretical arguments for the causes of data that are missing—for example, missing completely at random (MCAR), missing at random (MAR), or missing not at random (MNAR)›   
    \item Methods actually used for addressing missing data, if any‒descriptions of each primary and secondary outcome, including the total sample  and each subgroup, that includes the number of cases, cell means, standard deviations, and other measures that characterize the data used‒inferential statistics, including›   
    \item Results of all inferential tests conducted, including exact p values if null hypothesis significance testing (NHST) methods were used, and reporting the minimally sufficient set of statistics (e.g., dfs, mean square [MS] effect, MS error) needed to construct  the tests›   effect-size estimates and confidence intervals on estimates that correspond  to each inferential test conducted, when possible›   
    \item Clear differentiation between primary hypotheses and their tests–estimates,  secondary hypotheses and their tests–estimates, and exploratory hypotheses  and their test–estimates Statistics and Data Analysis (continued)‒complex data analyses—for example, structural equation modeling analyses (see also Table 7), hierarchical linear models, factor analysis, multivariate analyses, and so forth, including›   details of the models estimated›   
    \item Associated variance–covariance (or correlation) matrix or matrices›   
    \item Identification of the statistical software used to run the analyses (e.g., SAS PROC GLM or the particular R package)‒estimation problems (e.g., failure to converge, bad solution spaces), regression diagnostics, or analytic anomalies that were detected and solutions to those problems.‒other data analyses performed, including adjusted analyses, if performed, indicating those that were planned and those that were not planned (though not necessarily in the level of detail of primary analyses).
    \item Report any problems with statistical assumptions and/or data distributions that could affect the validity of findings

\end{itemize}


\section{Discussion}
\subsection{Support of Original Hypotheses}
Provide a statement of support or nonsupport for all hypotheses, whether primary or secondary, including‒distinction by primary and secondary hypotheses‒discussion of the implications of exploratory analyses in terms of both substantive findings and error rates that may be uncontrolled

\subsection{Similarity of Results}
Discuss similarities and differences between reported results and work of others

\subsection{Interpretation}

Provide an interpretation of the results, taking into account:

\begin{itemize}
    \item Sources of potential bias and threats to internal and statistical validity
    \item Imprecision of measurement protocols
    \item Overall number of tests or overlap among tests‒adequacy of sample sizes and sampling validity
\end{itemize}


\subsection{Generalizability}

Discuss generalizability (external validity) of the findings, taking into account:
\begin{itemize}
    \item target population (sampling validity)
    \item other contextual issues (setting, measurement, time; ecological validity)
\end{itemize}

\subsection{Implications}

Discuss implications for future research, program, or policy.

\section{ACKNOWLEDGMENTS}
Acknowledgments are placed before the references. Add information about grants, awards, or other types of funding that you have received to support your research. This information must be anonymized in the version of the paper you submit for review.  

\bibliographystyle{ACM-Reference-Format}
\bibliography{bibliography.bib}

\appendix


\end{document}
