
%%%% Proceedings format for most of ACM conferences (with the exceptions listed below) and all ICPS volumes.
\documentclass[acmsmall]{acmart}
\usepackage{multirow}
\usepackage{listings}
\usepackage{subcaption}
\usepackage[T1]{fontenc}

\def\BibTeX{{\rm B\kern-.05em{\sc i\kern-.025em b}\kern-.08emT\kern-.1667em\lower.7ex\hbox{E}\kern-.125emX}}
    
    
% Rights management information. 
% This information is sent to you when you complete the rights form.
% These commands have SAMPLE values in them; it is your responsibility as an author to replace
% the commands and values with those provided to you when you complete the rights form.
%
% These commands are for a PROCEEDINGS abstract or paper.
%\copyrightyear{2018}
%\acmYear{2018}
%\setcopyright{acmlicensed}
%\acmConference[Woodstock '18]{Woodstock '18: ACM Symposium on Neural Gaze Detection}{June 03--05, 2018}{Woodstock, NY}
%\acmBooktitle{Woodstock '18: ACM Symposium on Neural Gaze Detection, June 03--05, 2018, Woodstock, NY}
%\acmPrice{15.00}
%\acmDOI{10.1145/1122445.1122456}
%\acmISBN{978-1-4503-9999-9/18/06}

\acmYear{2021} \acmVolume{1} \acmNumber{1} \acmArticle{1} \acmMonth{1} 
% DOI
\acmDOI{10.1145/3418301}

% ISBN
\acmISBN{123-4567-24-567/08/06}

%Journal
\acmJournal{TOCE}

\acmYear{2021} \acmVolume{1} \acmNumber{1} \acmArticle{1} \acmMonth{1} 

\copyrightyear{2021}
\acmPrice{15.00}


\begin{document}


%From the APA JARS Specification
%Identify main variables and theoretical issues under investigation and the relationships between them.
%Identify the populations studied.
\title[JARS Qualitative Template for ACM TOCE]{JARS Qualitative Template for ACM TOCE}

%TOCE uses a double-blind review process. Omit author information and all identifying information in papers submitted for review. Please see TOCE’s Author Guidelines page (https://dl.acm.org/journal/toce/author-guidelines) for advice on anonymizing your paper.
\author{Madgique L. Worlds}
\email{madgique@gnome.edu}
\affiliation{\institution{Gnome College}}
\orcid{0000-0000-0000-0000}

\renewcommand{\shortauthors}{Worlds}

%Stolen from: https://tex.stackexchange.com/questions/74371/subdividing-structured-abstracts
\newcommand{\AbstractCategory}[1]{%
  \par\addvspace{.5\baselineskip}% adjust to suit
  \noindent\textbf{#1}\quad\ignorespaces
}

%
% The abstract is a short summary of the work to be presented in the article.
\begin{abstract}

Formatted abstracts are encouraged when possible and the ACM TOCE Quant template's abstract can be used here if applicable. Please do the following in the abstract:

\begin{itemize}
    \item State the problem/question/objectives under investigation.

    \item Indicate the study design, including types of participants or data sources, analytic strategy, main results/findings, and main implications/significance.

    \item Identify five keywords.
\end{itemize}

\textit{Guidance for Authors:} Consider including at least one keyword that describes the method and one that describes the types of participants or phenomena under investigation.  Consider describing your approach to inquiry when it will facilitate the review process and intelligibility of your paper. If your work is not grounded in a specific approach to inquiry or your approach would be too complicated to explain in the allotted word count, however, it would not be advisable to provide explication on this point in the abstract.
\end{abstract}



%
% The code below is generated by the tool at http://dl.acm.org/ccs.cfm.
% Please copy and paste the code instead of the example below.


\begin{CCSXML}
<ccs2012>
<concept>
<concept_id>10002950.10003705.10003708</concept_id>
<concept_desc>Mathematics of computing~Statistical software</concept_desc>
<concept_significance>500</concept_significance>
</concept>
<concept>
<concept_id>10010405.10010489</concept_id>
<concept_desc>Applied computing~Education</concept_desc>
<concept_significance>500</concept_significance>
</concept>
<concept>
<concept_id>10003120.10003121.10011748</concept_id>
<concept_desc>Human-centered computing~Empirical studies in HCI</concept_desc>
<concept_significance>300</concept_significance>
</concept>
<concept>
<concept_id>10011007.10011006.10011050.10011017</concept_id>
<concept_desc>Software and its engineering~Domain specific languages</concept_desc>
<concept_significance>300</concept_significance>
</concept>
</ccs2012>
\end{CCSXML}

\ccsdesc[500]{Mathematics of computing~Statistical software}
\ccsdesc[500]{Applied computing~Education}
\ccsdesc[300]{Human-centered computing~Empirical studies in HCI}
\ccsdesc[300]{Software and its engineering~Domain specific languages}


%
% Keywords. The author(s) should pick words that accurately describe the work being
% presented. Separate the keywords with commas.
\keywords{comma, separated}


%
% This command processes the author and affiliation and title information and builds
% the first part of the formatted document.
\maketitle

\textbf{Note for TOCE Submissions:} TOCE allows submissions of many varieties of study types. This template conforms to the generic template for qualitative studies (\url{https://apastyle.apa.org/jars/qual-table-1.pdf}). For meta-analysis of qualitative studies, see \url{https://apastyle.apa.org/jars/qual-table-2.pdf}. 

\vspace{3mm}
\textbf{Author's Note:} \textbf{Author's Note:} Provide acknowledgment and explanation of any special circumstances, including:

\begin{itemize}
    \item Registration information if the study has been registered
    \item Use of data also appearing in previous publications
    \item Prior reporting of the fundamental data in dissertations or conference papers
    \item Sources of funding or other support‒relationships or affiliations that may be perceived as conflicts of interest
    \item Previous (or current) affiliation of authors if different from location where the study  was conducted
    \item Contact information for the corresponding author
    \item Additional information of importance to the reader that may not be appropriately included in other sections of the paper
    \hyperlink{https://dl.acm.org/journal/toce/author-guidelines}{\color{blue}{TOCE’s Author Guidelines page}} for advice on anonymizing your paper.
\end{itemize}

\section{Introduction}

ACM TOCE strongly encourages authors to submit papers that adhere to the \hyperlink{https://apastyle.apa.org/jars/}{\color{blue}{American Psychological Association’s Journal Article Reporting Standards}}  (JARS).  Some papers that lie within the scope of ACM TOCE are not a good fit for these standards; we welcome such submissions. Whether or not your paper is a good fit for the standards, they can provide useful guidance to help make your paper more accessible to reviewers and the TOCE audience.
Here, we provide a JARS template for \hyperlink{https://apastyle.apa.org/jars/quant-table-1.pdf}{\color{blue}{general qualitative research designs}}. If your qualitative study is a meta-analysis, please see these JARS guidelines for \hyperlink{https://apastyle.apa.org/jars/qual-table-2.pdf}{\color{blue}{qualitative meta-analyses}}, which can be readily integrated into this template. See also the TOCE JARS templates for \hyperlink{https://apastyle.apa.org/jars/quantitative}{\color{blue}{quantitative}} and \hyperlink{https://apastyle.apa.org/jars/mixed-methods}{\color{blue}{mixed}} research designs.

Note that the ACM provides   \hyperlink{https://www.acm.org/binaries/content/assets/publications/consolidated-tex-template/acmart-master.zip}{\color{blue}{an article templates for LateX}}. Please download it and reference as appropriate as you write your paper. It contains detailed advice and guidance on a range of specific formatting issues not covered in this template.

JARS also recommends the following for the introduction:
\begin{itemize}
    \item Frame the problem or question and its context.
    \item Review, critique, and synthesize the applicable literature to identify key issues/debates/theoretical frameworks in the relevant literature to clarify barriers, knowledge gaps, or practical needs.
\end{itemize}

\textit{Guidance for Reviewers:} The introduction may include case examples, personal narratives, vignettes, or other
illustrative material.

\subsection{Description of Research Problem or Question}

\begin{itemize}
    \item Frame the problem or question and its context.
    \item Review, critique, and synthesize the applicable literature to identify key issues/debates/theoretical frameworks in the relevant literature to clarify barriers, knowledge gaps, or practical needs.
\end{itemize}

\textit{Guidance for Reviewers:} The introduction may include case examples, personal narratives, vignettes, or other illustrative material.

\subsection{Study Objectives/Aims/Research Goals}

\begin{itemize}
    \item State the purpose(s)/goal(s)/aim(s) of the study.
    \item State the target audience, if specific.
    \item Provide the rationale for fit of design used to investigate this purpose/goal (e.g., theory building, explanatory, developing understanding, social action, description, highlighting social practices).
    \item Describe the approach to inquiry, if it illuminates the objectives and research rationale (e.g., descriptive, interpretive, feminist, psychoanalytic, postpositivist, critical, postmodern, constructivist, or pragmatic approaches).
\end{itemize}

\textit{Guidance for Authors:} If relevant to objectives, explain the relation of the current analysis to prior articles/ publications.
\vspace{3mm}

\textit{Guidance for Reviewers} Qualitative studies often legitimately need to be divided into multiple manuscripts because of journal article page limitations, but each manuscript should have a separate focus. Qualitative studies tend not to identify hypotheses, but rather research questions and goals.

\section{Method}

\subsection{Research Design Overview}

\begin{itemize}
    \item Summarize the research design, including data-collection strategies, data-analytic
strategies, and, if illuminating, approaches to inquiry (e.g., descriptive, interpretive, feminist,
psychoanalytic, postpositivist, critical, postmodern, constructivist, or pragmatic approaches).
\item Provide the rationale for the design selected.
\end{itemize}

\textit{Guidance for Reviewers:}

\begin{itemize}
    \item Method sections can be written in a chronological or narrative format.
    \item Although authors provide a method description that other investigators should be able to follow, it is not required that other investigators arrive at the same conclusions but rather that the method description leads other investigators to conclusions with a similar degree of methodological integrity.
\item At times, elements may be relevant to multiple sections and authors need to organize what belongs in each subsection in order to describe the method coherently and reduce redundancy. For instance, the overview and the objectives statement may be presented
in one section.
\item Processes of qualitative research are often iterative versus linear, may evolve through the
inquiry process, and may move between data collection and analysis in multiple formats. As a result, data collection and analysis sections might be combined.
\item For the reasons above and because qualitative methods often are adapted and
combined creatively, requiring detailed description and rationale, an average qualitative
Method section typically is longer than an average quantitative Method section
\end{itemize}

\subsection{Study Participants or Data Sources}

\subsubsection{Researcher Description}

\begin{itemize}
    \item Describe the researchers’ backgrounds in approaching the study, emphasizing their
prior understandings of the phenomena under study (e.g., interviewers, analysts,
or research team).
\item Describe how prior understandings of the phenomena under study were managed
and/or influenced the research (e.g., enhancing, limiting, or structuring data collection
and analysis).
\end{itemize}

\textit{Guidance for Authors:} Prior understandings relevant to the analysis could include, but are not limited to, descriptions of researchers’ demographic/cultural characteristics, credentials, experience with phenomena, training, values, and/or decisions in selecting archives or material to
analyze.
\vspace{3mm}

\textit{Guidance for Reviewers:} Researchers differ in the extensiveness of reflexive self-description in reports. It may not be possible for authors to estimate the depth of description desired by reviewers
without guidance.

\subsubsection{Participants or Other Data Sources}

\begin{itemize}
    \item Provide the numbers of participants/documents/events analyzed.
\item  Describe the demographics/cultural information, perspectives of participants, or
characteristics of data sources that might influence the data collected.
\item  Describe existing data sources, if relevant (e.g., newspapers, internet, archive).
\item  Provide data repository information for openly shared data, if applicable.
\item  Describe archival searches or process of locating data for analyses, if applicable.
\end{itemize}


\subsubsection{Researcher–Participant Relationship}
\begin{itemize}
    \item Describe the relationships and interactions between researchers and participants
relevant to the research process and any impact on the research process (e.g., was
there a relationship prior to research, are there any ethical considerations relevant
to prior relationships).

\end{itemize}

\subsection{Participant Recruitment}
\subsubsection{Recruitment Process}

\begin{itemize}
    \item Describe the recruitment process (e.g., face-to-face, telephone, mail, email) and any
recruitment protocols.
\item  Describe any incentives or compensation, and provide assurance of relevant ethical
processes of data collection and consent process as relevant (may include institutional
review board approval, particular adaptations for vulnerable populations, safety monitoring).
\item  Describe the process by which the number of participants was determined in relation to the
study design.
\item  Provide any changes in numbers through attrition and final number of participants/sources
(if relevant, refusal rates or reasons for dropout).
\item  Describe the rationale for decision to halt data collection (e.g., saturation).
\item  Convey the study purpose as portrayed to participants, if different from the purpose stated.
\end{itemize}


\textit{Guidance for Authors/Reviewers:} The order of the recruitment process and the selection process and their contents may be determined in relation to the authors’ methodological approach. Some authors will determine a selection process and then develop a recruitment method based on those criteria. Other authors will develop a recruitment process and then select participants responsively in relation to evolving findings.
\vspace{3mm}

\textit{Guidance for Reviewers:} There is no agreed-upon minimum number of participants for a qualitative study. Rather, the author should provide a rationale for the number of participants chosen.

\subsubsection{Participant Selection}
\begin{itemize}
    \item Describe the participants/data source selection process (e.g., purposive sampling methods,
such as maximum variation; convenience sampling methods, such as snowball selection; theoretical sampling; diversity sampling) and inclusion/exclusion criteria.
\item  Provide the general context for the study (when data were collected, sites of data collection).
\item  If your participant selection is from an archived data set, describe the recruitment and selection process from that data set as well as any decisions in selecting sets of participants from that data set.
\end{itemize}


\textit{Guidance for Authors:} A statement can clarify how the number of participants fits with practices in the design at hand, recognizing that transferability of findings in qualitative research to other contexts
is based in developing deep and contextualized understandings that can be applied by readers rather than quantitative estimates of error and generalizations to populations.
\vspace{3mm}

\textit{Guidance for Authors/Reviewers:} The order of the recruitment process and the selection process and their contents may be determined in relation to the authors’ methodological approach. Some authors will determine a selection process and then develop a recruitment method based on those criteria. Other authors will develop a recruitment process and then select participants
responsively in relation to evolving findings.


\subsection{Data Collection}

\subsubsection{Data Collection/Identification Procedures}
\begin{itemize}
    \item State the form of data collected (e.g., interviews, questionnaires, media, observation).
\item  Describe the origins or evolution of the data-collection protocol.
\item  Describe any alterations of data-collection strategy in response to the evolving findings
or the study rationale.
\item  Describe the data-selection or data-collection process (e.g., were others present when data
were collected, number of times data were collected, duration of collection, context).
\item  Convey the extensiveness of engagement (e.g., depth of engagement, time intensiveness
of data collection).
\item  For interview and written studies, indicate the mean and range of the time duration in
the data-collection process (e.g., interviews were held for 75 to 110 min, with an average
interview time of 90 min).
\item  Describe the management or use of reflexivity in the data-collection process, as it
illuminates the study.
\item  Describe questions asked in data collection: content of central questions, form of questions
(e.g., open vs. closed).
\end{itemize}

\textit{Guidance for Reviewers:}

\begin{itemize}
    \item Researchers may use terms for data collection that are coherent within their research approach and process, such as "data identification," "data collection," or "data selection." Descriptions should be provided, however, in accessible terms in relation to the readership.
    \item  It may not be useful for researchers to reproduce all of the questions they asked in an interview, especially in the case of unstructured or semistructured interviews as questions are adapted to the content of each interview.
\end{itemize}

\subsubsection{Recording and Data Transformation}

Identify data audio/visual recording methods, field notes, or transcription processes used.

\subsection{Analysis}

\subsubsection{Data-Analytic Strategies}

\begin{itemize}
    \item Describe the methods and procedures used and for what purpose/goal.
\item Explicate in detail the process of analysis, including some discussion of the procedures
(e.g., coding, thematic analysis) following a principle of transparency.
\item Describe coders or analysts and their training, if not already described in the researcher
description section (e.g., coder selection, collaboration groups).
\item Identify whether coding categories emerged from the analyses or were developed a priori.
\item Identify units of analysis (e.g., entire transcript, unit, text) and how units were formed,
if applicable.
\item Describe the process of arriving at an analytic scheme, if applicable (e.g., if one was
developed before or during the analysis or was emergent throughout).
\item Provide illustrations and descriptions of the analytic scheme development, if relevant.
\item Indicate software, if used.
\end{itemize}

\textit{Guidance for Authors:} Provide rationales to illuminate analytic choices in relation to the study goals.
\vspace{3mm}

\textit{Guidance for Reviewers:} Researchers may use terms for data analysis that are coherent within their research approach and process (e.g., "interpretation," "unitization," "eidetic analysis," "coding").
Descriptions should be provided, however, in accessible terms in relation to the readership.

\subsubsection{Methodological Integrity}

\begin{itemize}
    \item Demonstrate that the claims made from the analysis are warranted and have produced findings with methodological integrity. The procedures that support methodological integrity (i.e., fidelity and utility) typically are described across the relevant sections of a paper, but they could be addressed in a separate section when elaboration or emphasis would be helpful. Issues of methodological integrity include the following:
    \begin{itemize}
        \item Assess the adequacy of the data in terms of its ability to capture forms of diversity most relevant to the question, research goals, and inquiry approach.
        \item Describe how the researchers’ perspectives were managed in both the data collection
        and analysis (e.g., to limit their effect on the data collection, to structure the analysis).
        \item Demonstrate that findings are grounded in the evidence (e.g., using quotes, excerpts, or
        descriptions of researchers’ engagement in data collection).
        \item Demonstrate that the contributions are insightful and meaningful (e.g., in relation to the
        current literature and the study goal).
        \item Provide relevant contextual information for findings (e.g., setting of study, information
        about participants, interview question asked is presented before excerpt as needed).
        \item Present findings in a coherent manner that makes sense of contradictions or disconfirming evidence in the data (e.g., reconcile discrepancies, describe why a conflict might exist in the findings).
    \end{itemize}

    \item Demonstrate consistency with regard to the analytic processes (e.g., analysts may use demonstrations of analyses to support consistency, describe their development of a stable perspective, interrater reliability, consensus) or describe responses to inconsistencies, as relevant (e.g., coders switching midway through analysis, an interruption in the analytic
    process). If alterations in methodological integrity were made for ethical reasons, explicate those reasons and the adjustments made.

    \item Describe how support for claims was supplemented by any checks added to the qualitative
    analysis. Examples of supplemental checks that can strengthen the research may include:
        \begin{itemize}
            \item transcripts/data collected returned to participants for feedback
            \item  triangulation across multiple sources of information, findings, or investigators
            \item  checks on the interview thoroughness or interviewer demands
            \item  consensus or auditing process
            \item  member checks or participant feedback on findings
            \item  data displays/matrices
            \item  in-depth thick description, case examples, or illustrations
            \item  structured methods of researcher reflexivity (e.g., sending memos, field notes, diary,
            logbooks, journals, bracketing)
            \item  checks on the utility of findings in responding to the study problem (e.g., an evaluation of
            whether a solution worked)
        \end{itemize}
\end{itemize}
 
\textit{Guidance for Reviewers:} Research does not need to use all or any of the checks (as rigor is centrally based in the iterative process of qualitative analyses, which inherently includes checks within the evolving, self-correcting iterative analyses), but their use can augment a study’s methodological integrity. Approaches to inquiry have different traditions in terms of using
checks and which checks are most valued.

\section{Findings/Results}

The findings/results section can be rather large in qualitative studies. It is recommended that subsections (and subsubsections) be used in context to the results being presented. Throughout the findings section:

\begin{itemize}
    \item Describe research findings (e.g., themes, categories, narratives) and the meaning and understandings that the researcher has derived from the data analysis.
    \item Present research findings in a way that is compatible with the study design.
    \item Demonstrate the analytic process of reaching findings (e.g., quotes, excerpts of data).
    \item Present synthesizing illustrations (e.g., diagrams, tables, models), if useful in organizing and conveying findings. Photographs or links to videos can be used.
\end{itemize}

\textit{Guidance for Authors:}
\begin{itemize}
    \item Findings presented in an artistic manner (e.g., a link to a dramatic presentation of findings) should also include information in the reporting standards to support the research presentation.
    \item Use quotes or excerpts to augment data description (e.g., thick, evocative description,
    field notes, text excerpts), but these should not replace the description of the findings of the analysis.
\end{itemize}

\textit{Guidance for Reviewers:}

\begin{itemize}
    \item The findings section tends to be longer than in quantitative papers because of the demonstrative rhetoric needed to permit the evaluation of the analytic procedure.
    \item Depending on the approach to inquiry, findings and discussion may be combined or a personalized discursive style might be used to portray the researchers’ involvement in the analysis.
    \item Findings may or may not include quantified information, depending upon the study’s goals, approach to inquiry, and study characteristics.
\end{itemize}

\section{Disussion}

It is suggested that the following be placed in subsections.

\subsection{Central Contributions to the Discipline}

Describe the central contributions and their significance in advancing disciplinary understandings.

\subsection{Types of Contributions Made by Findings}
Describe the types of contributions made by findings (e.g., challenging, elaborating on,
and supporting prior research or theory in the literature describing the relevance) and how
findings can be best utilized.

\subsection{Comparing Prior Theories and Research Findings}
Identify similarities and differences from prior theories and research findings.

\subsection{Alternative Explanations of Findings}
Reflect on any alternative explanations of the findings.

\subsection{Strengths and Limitations}
Identify the study’s strengths and limitations (e.g., consider how the quality, source, or types
of the data or the analytic processes might support or weaken its methodological integrity).

\subsection{Limitations on Transferability}
Describe the limits of the scope of transferability (e.g., what should readers bear in mind
when using findings across contexts).

\subsection{Ethical Dilemmas/Challenges}
Revisit any ethical dilemmas or challenges that were encountered, and provide related suggestions for future researchers.

\subsection{Implications for Future}
Consider the implications for future research, policy, or practice.

\vspace{3mm}
\textit{Guidance for Reviewers:} Accounts could lead to multiple solutions rather than a single one. Many qualitative approaches hold that there may be more than one valid and useful set of findings from a given data set.

\section{ACKNOWLEDGMENTS}
Acknowledgments are placed before the references. Add information about grants, awards, or other types of funding that you have received to support your research. This information must be anonymized in the version of the paper you submit for review.  

\bibliographystyle{ACM-Reference-Format}
\bibliography{bibliography.bib}

\appendix


\end{document}
